%% start of file `template.tex'.
%% Copyright 2006-2013 Xavier Danaux (xdanaux@gmail.com).
%
% This work may be distributed and/or modified under the
% conditions of the LaTeX Project Public License version 1.3c,
% available at http://www.latex-project.org/lppl/.


\documentclass[11pt,a4paper,sans]{moderncv}        % possible options include font size ('10pt', '11pt' and '12pt'), paper size ('a4paper', 'letterpaper', 'a5paper', 'legalpaper', 'executivepaper' and 'landscape') and font family ('sans' and 'roman')

% modern themes
\moderncvstyle{banking}                            % style options are 'casual' (default), 'classic', 'oldstyle' and 'banking'
\moderncvcolor{blue}                                % color options 'blue' (default), 'orange', 'green', 'red', 'purple', 'grey' and 'black'
%\renewcommand{\familydefault}{\sfdefault}         % to set the default font; use '\sfdefault' for the default sans serif font, '\rmdefault' for the default roman one, or any tex font name
%\nopagenumbers{}                                  % uncomment to suppress automatic page numbering for CVs longer than one page

% character encoding
\usepackage[utf8]{inputenc}                       % if you are not using xelatex ou lualatex, replace by the encoding you are using
%\usepackage{CJKutf8}                              % if you need to use CJK to typeset your resume in Chinese, Japanese or Korean

% adjust the page margins
\usepackage[scale=0.75]{geometry}
%\setlength{\hintscolumnwidth}{3cm}                % if you want to change the width of the column with the dates
%\setlength{\makecvtitlenamewidth}{10cm}           % for the 'classic' style, if you want to force the width allocated to your name and avoid line breaks. be careful though, the length is normally calculated to avoid any overlap with your personal info; use this at your own typographical risks...

\usepackage{import}

% personal data
\name{Pablo}{Margreff}
\title{Curriculum Vitae}                               % optional, remove / comment the line if not wanted
\address{Pelotas, Brasil}{}{}
\phone[mobile]{+55 53 8469 0251}
\phone[mobile]{+55 51 9774 9151}
\email{pmargreff@gmail.com}
%\photo[64pt][0.4pt]{picture}                       % optional, remove / comment the line if not wanted; '64pt' is the height the picture must be resized to, 0.4pt is the thickness of the frame around it (put it to 0pt for no frame) and 'picture' is the name of the picture file
%\quote{Some quote}                                 % optional, remove / comment the line if not wanted

% bibliography with mutiple entries
%\usepackage{multibib}
%\newcites{book,misc}{{Books},{Others}}
%----------------------------------------------------------------------------------
%            content
%----------------------------------------------------------------------------------
\begin{document}
%\begin{CJK*}{UTF8}{gbsn}                          % to typeset your resume in Chinese using CJK
%-----       resume       ---------------------------------------------------------
\makecvtitle

\small{Estudante do terceiro ano de Ciência da Computação na Universidade Federal de Pelotas, adepto a comunidade de software livre.}

\section{Experiência}

\vspace{6pt}

\begin{itemize}
\item{\cventry{Dezembro 2015 -- Atualmente}{Back-End}{Desenvolvedor}{Zero ou Um - Soluções digitais}{}{\vspace{3pt}Desenvolvimento e manutenção de plataformas web, com foco em soluções tecnológicas voltadas para a área de educação. Suporte a diversos tipos de sistemas, com foco em frameworks e tecnologias PHP.}}

\vspace{6pt}

\item{\cventry{Janeiro 2012 -- Fevereiro 2014}{Departamento Comercial}{Auxiliar Administrativo}{Unimed - Vales do Taquari e Rio Pardo}{}
{\vspace{3pt}Algumas atividades desempenhadas: conferência de contratos de pessoas fisicas e jurídicas e ocasionalmente atendimento ao cliente. Durante este período ajudei a planejar e implementar uma ideia para diminuir a quantidade de papel impresso gerado pelo setor. Com isso o departamento passou a imprimir mensalmente cerca de 6000 folhas a menos (50\% do total impresso), houve um ganho em tempo, diminuição no impacto ecológico, além de a médio-longo prazo diminuir os custos do departamento.}}

\vspace{6pt}

\end{itemize}

\section{Educação}

\vspace{5pt}

\subsection{Qualificações Acadêmicas}

\vspace{5pt}

\begin{itemize}

\item{\cventry{2014--Atual}{Bacharelado em Ciência da Computação}{Universidade Federal De Pelotas}{Pelotas}{\textit{}}{}}

\item{\cventry{2012--2013}{Técnico em Informática (Incompleto)}{Escola Estadual de Educação Profissional}{Estrela}{\textit{Realizados os três primeiros módulos}}{}}

\item{\cventry{2012--2012}{Curso de Administração}{Cooperativa de Ensino do Rio Grande Do Sul (SESCOOP)}{Lajeado}{\textit{}}{}}  

\end{itemize}

\vspace{2pt}

\subsection{Atividades e Projetos}

\vspace{5pt}

\begin{itemize}

\item{\textbf{Programa de Educação Tutorial (PET):} \textit{}

\vspace{3pt}

\small{O programa desenvolve habilidades em ensino, pesquisa e extensão, de maneira articulada e assim permitindo uma formação global, tanto do aluno colaborador quanto aos demais membros da comunidade acadêmica. Um exemplo de atividade desenvolvida é a semana acadêmica da computação, além de cumprir seu papel de disseminação de conhecimento para os 600 alunos dos cursos de ciência e engenharia da computação também atraí público externo à instituição. A inserção do grupo PET dentro do curso permite que estas capacidades se disseminem para os alunos do curso em geral, modificando e ampliando a perspectiva educacional de toda a comunidade.}}

\item{\textbf{Projeto de pesquisa: } \textit{'Interação humano-Computador através do reconhecimento de gestos usando o sensor Kinect'}

\vspace{3pt}

\small{O objetivo do projeto é estabelecer bases científicas e tecnológicas para o desenvolvimento de uma ferramenta que integre mapeamento, reconhecimento e aprendizagem de gestos em tempo real. Esta ferramenta deve ser capaz de interpretar a língua brasileira de sinais (LIBRAS), através da captura de movimentos e gestos, para tal, deverá ser utilizado o sensor Kinect da Microsoft para detectar os elementos gestuais. A ideia é que esta ferramenta pedagógica auxilie no processo de ensino/aprendizagem dos idiomas português e LIBRAS para crianças com deficiência auditiva.}}

\vspace{6pt}

\item{\textbf{Projetos Pessoais:}\textit{'Controlando dispositivos remotamente através do reconhecimento de gestos'}

\vspace{3pt}

\small{Permite o uso de gestos  capturados pelo Kinect para o disparar comandos em um módulo (via \textit{http requests} de controle de tomadas (feito com arduíno), podendo assim ligar e desligar dispositivos de maneira remota, no caso do nosso projeto uma cafeteira. :D

\textbf{>> Código Kinect - \href{https://github.com/pmargreff/KinectTracker}{https://github.com/pmargreff/KinectTracker}}

\textbf{>> Código cafeteira - \href{https://github.com/rsilveira65/smartcoffee}{https://github.com/rsilveira65/smartcoffee}}}
}

\end{itemize}
\section{Habilidades}

\vspace{6pt}

\begin{itemize}

\item \textbf{Linguagens de Programação:} C/C++, C\#, Java, Javascript PHP, MySQL, CSS, HTML5.
\item \textbf{Frameworks:} Materialize (CSS), AngularJS, NodeJS, Gulp, Grunt, Sass.
\item \textbf{Relacionados:} Domínio de orientação a objeto, e 
\vspace{6pt}

\item \textbf{Línguas:}
\\\textbf{>>} Inglês - Fala: Intermediário, Escrita: Intermediário, Compreensão: Intermediário. 
\\\textbf{>>} Espanhol - Fala: Básico, Escrita: Básico, Compreensão: Básico.  

\vspace{6pt}

\item \textbf{Outros:} Domínio em ferramentas de controle de versão (git), conhecimento em plataformas UNIX e entendimento sobre metodologias ageis.

\vspace{6pt}

\item \textbf{Habilidades interpessoais:} Proatividade, boa escrita, boa comunicação, entusiasmo em aprender coisas novas e disposição em ensinar coisas que domino.


\vspace{6pt}

\item \textbf{Interesses:} Docker, conceitos de linguagens de programação, inteligência artificial, programação funcional, desenvolvimento voltado a teste. 

\end{itemize}


\section{Eventos}

\vspace{6pt}
 
\begin{itemize}

\item{19º Semana Acadmica da Computação da Universidade Federal de Pelotas (2014) - Organizador}
\item{20º Semana Acadmica da Computação da Universidade Federal de Pelotas (2015)  - Organizador}
\item{16º Fórum Internacional do Software Livre (2015) - Participante}
\item{Code Arena / Hackathon By Universidade Federal de Pelotas(2015) - Organizador}
\end{itemize}

\section{Links Relevantes}

\vspace{6pt}
 
\begin{itemize}

\item{Blog Pessoal - \href{https://pmargreff.wordpress.com/}{https://pmargreff.wordpress.com/}}

\item{Artigos escritos na comunidade Viva o Linux  - \href{http://www.vivaolinux.com.br/~pmargreff/artigos/}{http://www.vivaolinux.com.br/\~{}pmargreff/artigos/}}

\item{Perfil no GitHub - \href{https://github.com/pmargreff}{https://github.com/pmargreff}}

\item{Perfil no Stack Overflow (PT) - \href{http://pt.stackoverflow.com/users/17688/pmargreff}{http://pt.stackoverflow.com/users/17688/pmargreff}}

\end{itemize}


% Publications from a BibTeX file without multibib
%  for numerical labels: \renewcommand{\bibliographyitemlabel}{\@biblabel{\arabic{enumiv}}}% CONSIDER MERGING WITH PREAMBLE PART
%  to redefine the heading string ("Publications"): \renewcommand{\refname}{Articles}
\nocite{*}
\bibliographystyle{plain}
\bibliography{publications}                        % 'publications' is the name of a BibTeX file

% Publications from a BibTeX file using the multibib package
%\section{Publications}
%\nocitebook{book1,book2}
%\bibliographystylebook{plain}
%\bibliographybook{publications}                   % 'publications' is the name of a BibTeX file
%\nocitemisc{misc1,misc2,misc3}
%\bibliographystylemisc{plain}
%\bibliographymisc{publications}                   % 'publications' is the name of a BibTeX file

%-----       letter       ---------------------------------------------------------

\end{document}


%% end of file `template.tex'.
