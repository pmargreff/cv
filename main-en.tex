%% start of file `template.tex'.
%% Copyright 2006-2013 Xavier Danaux (xdanaux@gmail.com).
%
% This work may be distributed and/or modified under the
% conditions of the LaTeX Project Public License version 1.3c,
% available at http://www.latex-project.org/lppl/.


\documentclass[11pt,a4paper,sans]{moderncv}        % possible options include font size ('10pt', '11pt' and '12pt'), paper size ('a4paper', 'letterpaper', 'a5paper', 'legalpaper', 'executivepaper' and 'landscape') and font family ('sans' and 'roman')

% modern themes
\moderncvstyle{banking}                            % style options are 'casual' (default), 'classic', 'oldstyle' and 'banking'
\moderncvcolor{blue}                                % color options 'blue' (default), 'orange', 'green', 'red', 'purple', 'grey' and 'black'
%\renewcommand{\familydefault}{\sfdefault}         % to set the default font; use '\sfdefault' for the default sans serif font, '\rmdefault' for the default roman one, or any tex font name
%\nopagenumbers{}                                  % uncomment to suppress automatic page numbering for CVs longer than one page

% character encoding
\usepackage[utf8]{inputenc}                       % if you are not using xelatex ou lualatex, replace by the encoding you are using
%\usepackage{CJKutf8}                              % if you need to use CJK to typeset your resume in Chinese, Japanese or Korean

% adjust the page margins
\usepackage[scale=0.75]{geometry}
%\setlength{\hintscolumnwidth}{3cm}                % if you want to change the width of the column with the dates
%\setlength{\makecvtitlenamewidth}{10cm}           % for the 'classic' style, if you want to force the width allocated to your name and avoid line breaks. be careful though, the length is normally calculated to avoid any overlap with your personal info; use this at your own typographical risks...

\usepackage{import}

% personal data
\name{Pablo}{Margreff}
\title{Resume}                               % optional, remove / comment the line if not wanted
\address{Pelotas, Brazil}{}{}
\phone[mobile]{+55 53 8469 0251}
\phone[mobile]{+55 51 9774 9151}
\email{pmargreff@gmail.com}
%\photo[64pt][0.4pt]{picture}                       % optional, remove / comment the line if not wanted; '64pt' is the height the picture must be resized to, 0.4pt is the thickness of the frame around it (put it to 0pt for no frame) and 'picture' is the name of the picture file
%\quote{Some quote}                                 % optional, remove / comment the line if not wanted

% bibliography with mutiple entries
%\usepackage{multibib}
%\newcites{book,misc}{{Books},{Others}}
%----------------------------------------------------------------------------------
%            content
%----------------------------------------------------------------------------------
\begin{document}
%\begin{CJK*}{UTF8}{gbsn}                          % to typeset your resume in Chinese using CJK
%-----       resume       ---------------------------------------------------------
\makecvtitle

\small{Undergraduate in computer science at Federal University of Pelotas.}

\section{Experience}

\vspace{6pt}

\begin{itemize}
\item{\cventry{December 2015 -- Currently}{Back-End}{Developer}{Zero ou Um - Digital Solutions}{}{\vspace{3pt}Developer and maintenance manager of web platforms, focused in solutions for the educational field. Work in PHP technologies and frameworks. Activities are mainly developed remotely.}}

\vspace{6pt}

\item{\cventry{January 2012 -- February 2014}{Commercial department}{Administrative Assistant}{Unimed - Vales do Taquari e Rio Pardo}{}
{\vspace{3pt}Some activities: Contract agreement validations, individuals or legal entities, costumer service. Co-participated as an developer and executor of the idea to decrease the paper consume of the department. The result of the application was the saving of six thousand sheet in a month (50\% of total print out) also it helped to reduce the ecological impact, and in medium-long term we cut off costs of the department.}}

\vspace{6pt}

\end{itemize}

\section{Education}

\vspace{5pt}

\subsection{Academic}

\vspace{5pt}

\begin{itemize}

\item{\cventry{2014--Currently}{Bachelor Of Computer Science}{Federal University of Pelotas}{Pelotas - Brazil}{\textit{}}{}}

\item{\cventry{2012--2013}{Computer technician (Not finished)}{State school of professional education}{Estrela - Brazil}{\textit{The three first semesters}}{}}

\item{\cventry{2012--2012}{Administration course}{Rio Grande Do Sul education cooperative (SESCOOP)}{Lajeado - Brazil}{\textit{}}{}}  

\end{itemize}
\section{Skills}

\vspace{6pt}

\begin{itemize}

\item \textbf{Programming languages:} C/C++, C\#, Java, PHP, MySQL, CSS, HTML5, Javascript.
\item \textbf{Frameworks:} Materialize (CSS), AngularJS, NodeJS, Gulp, Sass.
\item \textbf{Languages:}
\\\textbf{>>} English - Speaks: Intermediate, Writing: Beginner, Reading: Intermediate. 
\\\textbf{>>} Spanish - Speaks: Intermediate, Writing: Beginner, Reading: Intermediate. 
\\\textbf{>>} Portuguese - Native. 
\item \textbf{Others:} Strong skills in oriented-object pattern, and projects patterns, version control knowledge (git), experience with UNIX.
\item \textbf{Personal skills:} Pro activity, Blog Writer (most times in Portuguese), Passion in learn new things and teach them.
\end{itemize}

\vspace{2pt}

\section{Extracurricular Activities}

\vspace{5pt}

\begin{itemize}

\item{\textbf{Tutorial Educational Program (PET):} \textit{}

\vspace{3pt}

\small{The program helps develop skills in education, search and leadership. An example is the academic week, which it's made for more than 6 hundred of student with talks, week-courses, and debates to have a better computer science/engineering, the talks and week-course also help the students to learn things usually don't teach on college.}}

\item{\textbf{Search project:} \textit{'Motion recognition using Kinect for a computer-human interaction:'}

\vspace{3pt}

\small{The goal of project is to build a scientific and technological base in order to develop a recognition tool for LIBRAS (Brazilian Sign Language). It uses some artificial intelligence to constantly increasing the percentage of assertiveness in real time. The final goal is a pedagogic tool for teachers, where it will translate the LIBRAS,  easing the process of teaching. For all this, We have used Microsoft Kinect V2 to get the motion and WEKA library to classify the motion. }}

\vspace{6pt}

\item{\textbf{Personal projects:}\textit{'Using motion recognition to access remote devices.'}

\vspace{3pt}

\small{The work consists in the capture of a motion that will be sent to a plug control (built with Arduino) that allows to turn on/off remotely any device plugged on the special plug (by \textit{http requests}). The first implementation was a coffee-machine.

\textbf{>> Kinect code - \href{https://github.com/pmargreff/KinectTracker}{https://github.com/pmargreff/KinectTracker}}

\textbf{>> Coffee-machine code - \href{https://github.com/rsilveira65/smartcoffee}{https://github.com/rsilveira65/smartcoffee}}}
}

\end{itemize}

\section{Relevant links}

\vspace{6pt}
 
\begin{itemize}

\item{Personal Blog - \href{https://pmargreff.wordpress.com/}{https://pmargreff.wordpress.com/}}

\item{Articles on Viva o Linux (Portuguese) - \href{http://www.vivaolinux.com.br/~pmargreff/artigos/}{http://www.vivaolinux.com.br/\~{}pmargreff/artigos/}}

\item{GitHub Profile - \href{https://github.com/pmargreff}{https://github.com/pmargreff}}

\item{Stack Overflow Profile (Portuguese) - \href{http://pt.stackoverflow.com/users/17688/pmargreff}{http://pt.stackoverflow.com/users/17688/pmargreff}}

\end{itemize}


% Publications from a BibTeX file without multibib
%  for numerical labels: \renewcommand{\bibliographyitemlabel}{\@biblabel{\arabic{enumiv}}}% CONSIDER MERGING WITH PREAMBLE PART
%  to redefine the heading string ("Publications"): \renewcommand{\refname}{Articles}
\nocite{*}
\bibliographystyle{plain}
\bibliography{publications}                        % 'publications' is the name of a BibTeX file

% Publications from a BibTeX file using the multibib package
%\section{Publications}
%\nocitebook{book1,book2}
%\bibliographystylebook{plain}
%\bibliographybook{publications}                   % 'publications' is the name of a BibTeX file
%\nocitemisc{misc1,misc2,misc3}
%\bibliographystylemisc{plain}
%\bibliographymisc{publications}                   % 'publications' is the name of a BibTeX file

%-----       letter       ---------------------------------------------------------

\end{document}


%% end of file `template.tex'.
